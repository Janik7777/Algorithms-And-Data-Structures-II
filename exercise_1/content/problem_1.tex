\subsection{Problem 1}

\begin{enumerate}
    \item \textbf{Give the important differences between a normal priority queue and an addressable priority queue.}
    
    Normal priority queues maintain a set $M$ of elements offering the following operations: \\
    $build(\{e_1,...,e_n\}): M := \{e_1,...,e_n\}$ \\
    $insert(e): M := M \cup \{e\}$ \\
    $min: return \; min \; M$ \\
    $deleteMin: e := min \; M; M := M \backslash \{e\}; return \; e$

    Addressable priority queues additionally support operations on arbitrary elements addressed by an element handle $h$: \\
    $insert$: As before but return a handle to the element inserted. \\
    $remove(h)$: Remove the element specified by handle $h$. \\
    $decreaseKey(h: Handle, k: Key)$: Decrease the key of the element specified by handle $h$ to key $k$. \\
    $merge(Q_2): Q_1:= Q_1 \cup Q_2; Q_2 := \emptyset$.

    \item \textbf{Compare the running time of a merge operation for pairing heaps and binary heaps.}
    
    The \textit{Pairing Heap} can merge another \textit{Pairing Heap} in constant time $O(1)$ as only the minPtr is eventually updated and the addressable priority queue to be merged is simply attached to the forest.
    
    The \textit{Binary Heap} has a complexity of $O(n+m)$ as it throws the n elements of the first and the $m$ elements of the second \textit{Binary Heap} together and then calls $build((\{e_1,...,e_{n+m}\}))$ again.
   
\end{enumerate}
