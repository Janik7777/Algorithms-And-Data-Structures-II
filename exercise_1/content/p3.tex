\subsection{Problem 3}

\textbf{For a very big festival, you are responsible for the bar. For this task, you designed a bar robot (aka Bender) that mixes excellent cocktails. The ingredients are stored in big boxes. But exactly here is the problem: you have to make sure that none of the boxes runs empty. On the other hand, you also want to enjoy the evening and not constantly check all of the boxes. To deal with this problem, there is only one solution: a refill display! Unfortunately, the only displays available are the ones that can display one line only.
It is clear that this line should display the most urgently need ingredient. The goal is to design an algorithm that keeps the display up to date.}

\begin{enumerate}
    \item \textbf{As a base for you algorithm, think about a data structure to efficiently implement your algorithm. Assume that the number of ingredients for a specific cocktail is significantly smaller than the number of ingredients totally available.}

\item \textbf{Design a function MixDrink( recipe ), that operates on your data structure. Give Pseu- docode. Your data structure has to be updated! Other functions of the robot don’t have to be “programmed”.}

\item \textbf{When a box is exchanged, your data structure has to be updated as well. Describe what consequences the exchange has on your data structure.}

\end{enumerate}
